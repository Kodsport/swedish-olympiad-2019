\documentclass[a4paper,12pt,oneside]{amsbook}
\usepackage[T1]{fontenc}
\usepackage{textcomp}
%\usepackage{euler}
%\usepackage{garamond}

\usepackage{newcent}
%\fontfamily{fxm}\selectfont
\usepackage{fancyhdr}
\pagestyle{fancy}
% Detta paket skoter grafiken
\usepackage[dvips]{graphicx}
%\usepackage[pdftex]{graphicx}
\usepackage{tabularx}
\usepackage{amsthm}
\usepackage{amsmath}
\usepackage[swedish]{babel}
% For bortkommentering
\usepackage{verbatim}
% For att slippa indentering av Sections och SubSections
\usepackage{indentfirst}

% Har stalls paragrafindentering och avstand mellan stycken och rader mm
\setlength{\parindent}{0pt}
\setlength{\parskip}{6pt}
\setlength{\baselineskip}{12pt}
% Har definieras sidan
\setlength{\voffset}{-15mm}
\setlength{\hoffset}{-5.5mm}

\setlength{\topmargin}{0mm}
\setlength{\headheight}{10mm}
\setlength{\headsep}{8mm}
\setlength{\footskip}{16mm}

\setlength{\evensidemargin}{0mm}
\setlength{\oddsidemargin}{0mm}

\setlength{\marginparsep}{0mm}
\setlength{\marginparwidth}{0mm}

\setlength{\textwidth}{170mm}
\setlength{\textheight}{225mm}
\setlength{\paperwidth}{210mm}
\setlength{\paperheight}{297mm}
\setlength{\headwidth}{170mm}
% Huvud och Fot
%\fancypagestyle{}{
\fancyhf{}\fancyhead[c]{\small{\textit{Svar och r�ttningsanvisningar
      till Programmeringsolympiadens skolkval 2019}}}
\fancyfoot[c]{\thepage}
\renewcommand{\headrulewidth}{0.4pt}
\renewcommand{\footrulewidth}{0.4pt}
\newenvironment{lista}
{\begin{itemize}
\setlength{\parindent}{0pt}
\setlength{\itemsep}{6pt}}
{\end{itemize}}
% Uppgifter
\newcounter{probnr}
\newenvironment{tal}{%
\begin{list}
%{\textbf{\arabic{section}.\arabic{probnr}}} {\usecounter{probnr}
{\textbf{\arabic{probnr}}} {\usecounter{probnr}
\setlength{\leftmargin}{0mm}
\setlength{\rightmargin}{0mm}
\setlength{\labelwidth}{-1mm}
\setlength{\labelsep}{1mm}}
\setlength{\itemsep}{6pt}
}{\end{list}}
% Definition av avsnitt: FAKTA, EXEMPEL, PASTAENDE
\newtheorem{fakta}{Fakta}
\newtheorem{exempel}{Exempel}
\newtheorem{problem}{Problem}
%
\newtheoremstyle{test}% NAME
{20pt}      % ABOVESPACE
{10pt}      % BELOWSPACE
{\sffamily} % BODYFONT
{0pt}       % INDENT
{\scshape}  % HEADFONT
{}          % HEADPUNCT
{\newline}  % HEADSPACE
{}          % CUSTOM-HEAD-SPEC

\theoremstyle{test}
\newtheorem{program}{Program}
\newcommand{\sv}[1]{\textsc{#1}}            % Sma versaler
\newcommand{\fe}[1]{\textbf{#1}}            % FET
\newcommand{\ku}[1]{\textit{#1}}            % KURSIV
\newcommand{\cu}[1]{\texttt{#1}}            % Courier
\newcommand{\sk}[1]{\texttt{#1}}            % Courier
\newcommand{\rubrik}[1]{\begin{center}\sf\huge{#1}\normalsize\rm\end{center}}
\begin{document}
%\DeclareGraphicsExtensions{.jpg,.pdf,.mps,.png,.eps}


\specialsection*{Svar och r�ttningsanvisningar}
\thispagestyle{fancy}
\lhead{}
\begin{itemize}
%\setlength\itemsep{0.2cm}
\item L�s igenom \textit{t�vlingsreglerna}.
\item Programmen tas i tur och ordning in i editorn och kompileras.
Uppst�r kompileringsfel betraktas programmet som felaktigt och l�sningen
ges $0$ po�ng.
\item Programmet k�rs med givna indata enligt nedan. Alternativt, om eleven f�rberett programmet f�r det, kan de bifogade indatafilerna anv�ndas ist�llet.
\item Varje testfall
  med korrekt svar ger 1 po�ng.
\item
Totalt kan man
  allts� f� 5 po�ng f�r varje uppgift.
\item Om exekveringstiden f�r ett testexempel, k�rt p� en modern dator,
\textit{�verskrider 3 sekunder} betraktas k�rningen av testexemplet som felaktigt.
\item Det kan vara viktigt att programmet k�rs i en milj� liknande den som
programmet utvecklats i, samma version av kompilator eller
interpretator.
\item Vid problem i samband med r�ttningen �r det viktigt att det sunda
f�rnuftet f�r r�da!
\item Ett f�rslag till r�ttningsprocedur kan vara att l�ta eleven
sitta vid datorn.
\end{itemize}

\vspace{2cm}

\subsection*{Uppgift 1 -- Kuber}
~\\
~\\
{\tt 
\begin{tabular}{||l||c||c||}\hline\hline
& {\fe{Indata}} & \fe{Utdata} \\ \hline \hline
\fe{Test 1} & 10 & 3025 \\ \hline
\fe{Test 2} & 5 & 225 \\ \hline
\fe{Test 3} & 31 & 246016 \\ \hline
\fe{Test 4} & 14 & 11025 \\ \hline
\fe{Test 5} & 23 & 76176 \\ \hline\hline
\end{tabular}
}
%$$


\subsection*{Uppgift 2 -- Studschiffer}
~\\
~\\
{\tt 
\begin{tabular}{||l||c|c|l||l||}\hline\hline
& \multicolumn{3}{c||}{\fe{Indata}} & \fe{Utdata} \\ 
& $N$ & $M$ & Krypterad text & Meddelande \\ \hline \hline
\fe{T1}& 2 & 8 &\texttt{HJAESN}&\texttt{HEJSAN}\\ \hline                           
\fe{T2}& 5 & 10 &\texttt{VGAITDPKEN}&\texttt{VADKNEPIGT}\\ \hline                            
\fe{T3}& 4 & 20 &\texttt{IADANKNOATGNAKDTEVE}&\texttt{INGENKANAVKODADETTA}\\ \hline 
\fe{T4}& 11 & 7 &\texttt{JIANFGIAHSLOEEPSPINAVSTAT}&\texttt{JAGHOPPASATTVISESIFINALEN}\\ \hline          
\fe{T5}& 5 & 14 &\texttt{GRTUSRKODKEYAASILDPTISAATTTR}&\texttt{GRATTISDUKLARADESISTAKRYPTOT}\\ \hline 
\hline
\end{tabular}
}

\newpage
\subsection*{Uppgift 3 -- Renoveringen}
~\\
{\tt 
\begin{tabular}{||l||c|c|l|l||c|l||}\hline\hline
& \multicolumn{4}{c||}{\fe{Indata}} & \multicolumn{2}{c||}{\fe{Utdata}} \\ 
& $N$ & $M$ & $x_1, x_2, ..., x_N$ & $y_1, y_2, ..., y_M$  &Antal & L�ngder \\ \hline \hline
\fe{Test 1}&3 & 1 & 5 7 8 & 7 & 2 & 5 8 \\ \hline 
\fe{Test 2}&6 & 3 & 1 2 1 4 100 2 & 1 100 1 & 3 & 2 2 4 \\ \hline 
\fe{Test 3}&6 & 10 &  2 3 12 8 98 98 & 2 1 2 1 11  &  1 &98  \\ 
&&&& 9 99 97 27 28 &&\\ \hline  
\fe{Test 4}&15 & 15 & 20 21 19 27 100 & 20 18 17 18 11& 14 & 19 21 23 23 27   \\ 
&&& 28 29 23 23 67  & 1 16 18 14 17  && 28 29 43 61 67     \\ 
&&& 43 97 67 61 78 & 3 9 7 7 9 & & 67 78 97 100    \\ \hline  
\fe{Test 5}&15 & 10 & 98 99 90 89 76  & 100 1 91 78 68 & 7 &  15 63 69 75    \\ 
&&& 78 75 69 63 65 & 44 42 8 22 14&&  76 89 98   \\ 
&&& 40 43 15 9 22 &               & &     \\ \hline  
\hline
\end{tabular}
}


\subsection*{Uppgift 4 -- Multationer}
~\\
{\tt 
\begin{tabular}{||l||l|l||l|l|l||}\hline\hline
& \multicolumn{2}{c||}{\fe{Indata}} & \multicolumn{3}{c||}{\fe{Utdata}} \\ 
& Start & M�l & Mult 1 & Mult 2 & Mult 3 \\ \hline \hline
\fe{Test 1}&\texttt{BCBAC}&\texttt{ACACC}      & A -> C & B -> A &             \\ \hline                           
\fe{Test 2} &\texttt{ABC}&\texttt{CCAACAABC}   & A -> CAA & A -> CAA &         \\ \hline                            
\fe{Test 3} &\texttt{ACAB}&\texttt{ACABCACACA} & B -> BCA & B -> BCA & B -> BCA\\ \hline 
\fe{Test 4} &\texttt{CCA}&\texttt{CACACB}      & A -> B & C -> CA & B -> CB    \\ \hline          
\fe{Test 5} &\texttt{AB}&\texttt{CBBCBABCBA}   & A -> CB & B -> BCA & C -> CB  \\ \hline 
\hline
\end{tabular}
}


\subsection*{Uppgift 5 -- Robott�vling}
~\\
{\tt 
\begin{tabular}{||l||c|l|l||c|l||}\hline\hline
& \multicolumn{3}{c||}{\fe{Indata}} & \multicolumn{2}{c||}{\fe{Utdata}} \\ 
& $N$ & $r_1, r_2, ..., r_N$ & $c_1, c_2, ..., c_N$  &Min & Max \\ \hline \hline
\fe{Test 1}&3 &   1 4 5 & 3 5 5                               &22$^*$ &27 \\ \hline 
\fe{Test 2}&10 &  2 3 5 1 5 5 4 4 3 5 & 1 1 2 3 2 2 5 3 3 2   &132$^*$ &214 \\ \hline 
\fe{Test 3}&3 &   4 5 5 & 3 5 5                               &22 &37 \\ \hline
\fe{Test 4}&9 &   5 5 3 1 1 5 3 2 3 & 3 3 5 1 2 5 1 4 5       &103 &187 \\ \hline
\fe{Test 5}&10 &  5 3 5 2 5 5 2 3 3 5 & 5 5 5 4 5 4 4 5 1 3   &137 & 327 \\ \hline
\hline
\end{tabular}
}\\
$^*$OBS! Min-v�rdet beh�ver inte vara korrekt f�r att ge po�ng p� test 1 och 2.


\subsection*{Uppgift 6 -- Avslutningsceremonin}
~\\
{\tt 
\begin{tabular}{||l||l|l||l|l|l||}\hline\hline
& \multicolumn{2}{c||}{\fe{Indata}} & \fe{Utdata} \\ 
& Rad & Maxavst�nd & \\ \hline \hline
\fe{Test 1} & CACBACAABABBCBC               & 1 & 8\\ \hline                           
\fe{Test 2} & AABAABBAAABBBABABBAAABABBABAAB& 2 & 21\\ \hline                            
\fe{Test 3} & BABABABABABBAABABABABABABABABA& 2 & 20\\ \hline 
\fe{Test 4} & AABCDCDDABDBDACDBBDBACADCDBABC& 2 & 14\\ \hline          
\fe{Test 5} & CBABABABABABABABABABABABABADAD& 2 & 18\\ \hline 
\hline
\end{tabular}
}

\end{document}
