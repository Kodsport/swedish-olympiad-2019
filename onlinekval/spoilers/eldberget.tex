\section*{D - Eldberget}

Ifall $k=0$ är uppgiften att helt enkelt hitta kortaste vägen i en \href{https://sv.wikipedia.org/wiki/Graf_(grafteori)}{graf},
där en ruta (som inte har eld på sig) i rutnätet är en nod och en kant går mellan intilliggande rutor.
Standardalgoritmen för att hitta kortaste vägen mellan ett par av punkter i en graf är ``bredden-först-sökning'', BFS, och den fungerar alldeles utmärkt här.
Algoritmen går ut på att först hitta alla noder på avstånd 1 till startnoden (alla dess grannar),
sedan alla noder på avstånd 2 (alla grannar till noderna på avstånd 1), sedan alla noder på avstånd 3
(alla grannar till noderna på avstånd 2 som vi inte redan gett ett avstånd till), och så vidare.
Man kommer alltså att "processa" alla noder i ordning efter deras avstånd till startnoden.
Ett smidigt sätt att implementera detta på är att ha en kö med alla noder som ska processas.
När vi processar en nod så lägger vi in dess grannar (de som inte redan processats) längst bak i kön,
tillsammans med de avståndet från startnoden de är på.
När vi kommer till slutnoden (rutan längst ned i högra hörnet) så skriver vi bara ut avståndet vi är på.

När $k>0$ kan vi fortfarande göra BFS, men vi måste vara lite smartare.
Man kan tänka att vi kan vara vid en viss nod på $k$ olika sätt, beroende på hur många eldar vi hittills gått igenom.
Vi kan se det som att vi gör $k$ stycken kopior av varje nod, en för varje antal eldar man gått genom innan, och gör en BFS på den nya grafen vi får.
Man kan även tänka på det som att vi lägger till en till dimension till rutnätet.
För en ruta utan eld på så säger vi att den har kanter som går till alla dess grannar i samma lager i det 3-dimensionella rutnätet,
och för en ruta med eld så går det kanter till alla grannar i lagret ovanför (eftersom man gått igenom en mer eld).

Nu kör vi BFS:en som vanligt på den här grafen. När vi processar en nod med eld som är på översta lagret (d.v.s har redan gått genom $k$ stycken eldar) så ignorerar vi den,
då vi inte kan fortsätta gå någonstans därifrån. Ifall vi aldrig i BFS kommer till slutnoden är svaret $-1$;

Exempellösning i c++:
\lstinputlisting{eldberget.cpp}
