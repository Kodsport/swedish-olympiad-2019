\section*{G - Husbygge}

Det finns många sätt att lösa Husbygge på, och vi har inte fått in någon lösning som är strikt bättre än alla andra på alla testfall (den ihopkombinerade bästa deltagarlösningen kommer från 5 olika personer).
Med det sagt kommer här en förklaring av ungefär hur domarlösningen fungerar. Domarlösningen gör inte så stor skillnad mellan de olika testfallen, utan det är ungefär samma lösning på alla fast med lite olika parametrar. Undantaget är testfall 3 som kan lösas fullständigt med dynamisk programmering.

Det brukar kunna vara en förvånansvärt bra idé att skriva en ganska enkel girig lösning som är väldigt snabb, och sen köra den en massa gånger och ta det bästa svaret. Om man har tid över kan man ta lösningen och göra slumpmässiga förändringar och se om svaret blir bättre, tills tiden tar slut.

Först och främst: hur räknar man en lösnings poäng på bästa sätt? Det som tar lång tid är att hitta den närmsta grannen till varje person. Vi kan kolla alla par av hus, men det tar $O(k^2)$ vilket blir väldigt långsamt för de stora testfallen. Dessutom vill vi kunna räkna poäng för många lösningar, så det finns mycket att tjäna på att snabba upp den här delen av lösningen. Istället kan vi göra en bredden-först-sökning från varje hus och avbryta när vi träffar på första grannen. Det kanske verkar som om det här är $O(nmk)$ i värsta fall, men eftersom grafen är ett rutnät och sökningarna avbryts tidigt blir det i stället $O(nm)$ -- varje ruta kan besökas av högst $8$ BFS:er, två per kvadrant sett från den rutan. Om den hade besökts av fler än så hade någon av dessa BFS:er hittat en annan BFS:s startpunkt innan den nått rutan vi kollat på.

Grundidé: placera ut husen girigt på rutor med störst $a_{ij}$ först. När du har placerat ut ett hus, förbjud alla rutor inom avstånd $d$ från huset. På det här sättet får man en lösning där alla hus är ganska värdefulla, men samtidigt med avstånd $\geq d$ från alla andra hus (det kan bli så att de ``tillåtna'' rutorna tar slut, i så fall knökas resten av personerna in i övre vänstra hörnet). Den här lösningen tar faktiskt bara $O(k+d^2)$ tid, men det tar $O(\min(nm,k^2))$ tid att räkna poängen.

Den här lösningen körs en massa gånger för olika värden på $d$ och lite slumpad ordning på rutorna. Sedan är det så klart en massa detaljer om hur man får det här att gå snabbt etc.

Domarlösningen är ganska ful och lång. Här är Måns Python-lösning som är en tredjedel så lång och får mer poäng, och verkar använda en liknande girig strategi:

\lstinputlisting{husbygge.py}
