\problemname{Betygsättning}
Pelle är programmeringslärare på Pelles Optimeringsskola (PO-skolan).
Han håller nu på att sätta betyg på sina elever i kursen Optimering $1$.

Betygsättning går till på följande vis.
Totalt finns det $x$ \texttt{A}-kriterier, $y$ \texttt{C}-kriterier och $z$ \texttt{E}-kriterier som används.
För att få betyget \texttt{E} måste man uppfylla samtliga \texttt{E}-kriterier.
För att få betyget \texttt{C} måste man uppfylla samtliga \texttt{C}- och \texttt{E}-kriterier.
För att få betyget \texttt{A} måste man uppfylla samtliga \texttt{A}-, \texttt{C}- och \texttt{E}-kriterier.

Dessutom finns det två speciella betyg.
Om man uppfyller alla \texttt{E}-kriterier och minst hälften av \texttt{C}-kriterierna får man ett \texttt{D}.
Om man uppfyller alla \texttt{E}- och \texttt{C}-kriterier och minst hälften av \texttt{A}-kriterierna får man ett \texttt{B}.

Pelle tycker det är väldigt jobbigt att sätta betyg, och behöver din hjälp.
Skriv ett program som tar emot antalet \texttt{A}-, \texttt{C}- och \texttt{E}-kriterier en viss elev har uppfyllt och skriver ut vilket betyg eleven ska ha.
Du kan anta att eleven alltid fick minst \texttt{E} i kursen.

\section*{Indata}
På första raden står tre heltal $x, y, z$ ($1 \leq x, y, z \leq 30$), antalet \texttt{A}-, \texttt{C}- och \texttt{E}-kriterier som finns.
På den andra raden står tre heltal $x', y', z'$ ($0 \leq x' \leq x$, $0 \leq y' \leq y$ och $0 \leq z' \leq z$), antalet \texttt{A}-, \texttt{C}- och \texttt{E}-kriterier som eleven har uppfyllt.

\section*{Utdata}
Skriv ut en bokstav: \texttt{A}, \texttt{B}, \texttt{C}, \texttt{D}, eller \texttt{E}.

\section*{Poängsättning}
Din lösning kommer att testas på en mängd testfallsgrupper.
För att få poäng för en grupp så måste du klara alla testfall i gruppen.

\noindent
\begin{tabular}{| l | l | l |}
\hline
Grupp & Poängvärde & Gränser \\ \hline
$1$     & $40$        &  Betyget är \texttt{A}, \texttt{C} eller \texttt{E} \\ \hline 
$2$     & $60$        &  Inga ytterligare begränsningar. \\ \hline
\end{tabular}

\section*{Förklaring av exempelfall}
I exempelfall $1$ uppfyller eleven alla kriterier utom ett \texttt{A}-kriterium. Eleven får därför ett \texttt{B}.

I exempelfall $2$ uppfyller eleven alla \texttt{E}-kriterier men inte hälften av \texttt{C}-kriterierna. Eleven får därför ett \texttt{E}.

