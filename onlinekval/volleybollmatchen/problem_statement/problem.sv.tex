\problemname{Volleybollmatchen}
Du har blivit inhyrd av PO-Volley (Påhittade Organisationen för Volleyboll),
för att hålla koll på poängen i en volleybollmatch mellan de två lagen Algoritmikerna och Bäverbusarna.

I volleyboll spelar man bäst av $3$ set, det vill säga det laget
som först vunnit $2$ set vinner matchen. I de två första setten gäller först
till $25$, och ifall det blir ett tredje set spelar man då först till $15$.
Man måste vinna med $2$ poäng, så det är alltså det lag som först har
minst $25$ resp $15$ poäng, och som dessutom har $2$ poäng mer än motståndaren
som vinner det settet. Givet vilket lag som vinner varje boll, skriv ut resultatet av matchen.

\section*{Indata}
Den första raden innehåller ett heltal $N$ ($1 \le N \le 200$).
Därefter följer en $N$ tecken lång sträng bestående av ``\texttt{A}'' och ``\texttt{B}'', som beskriver en hel match.
Den $i$:te bokstaven är ``\texttt{A}'' ifall Algoritmikerna vinner den $i$:te bollen, och ``\texttt{B}'' ifall lag Bäverbusarna vinner den $i$:te bollen.

Matchen kommer att vara fullständig, d.v.s. något av lagen kommer ha kommit upp i $2$ poäng i slutet, och inga extra bollar kommer ha spelats.

\section*{Utdata}
Skriv ut en rad med två heltal. Det första talet ska vara antalet set Algoritmikerna vann
och det andra talet ska vara antalet set Bäverbusarna vann.

\section*{Poängsättning}
Din lösning kommer att testas på en mängd testfallsgrupper.
För att få poäng för en grupp så måste du klara alla testfall i gruppen.

\noindent
\begin{tabular}{| l | l | l |}
\hline
Grupp & Poängvärde & Gränser \\ \hline
$1$    & $42$          &  Alla set kommer vinnas på exakt $25$, $25$ och $15$ poäng respektive, det vill säga det kommer aldrig bli $24$ lika resp. $14$ lika\\ \hline 
$2$    & $58$          &  Inga ytterligare begränsningar \\ \hline
\end{tabular}

\section*{Förklaring av exempelfall}
I det första exempelfallet vinner lag A första settet med $25-0$,
lag B vinner andra settet med $0-25$, och lag A vinner det avgörande
tredje settet med $15-0$.
Resultatet av matchen blir alltså $2-1$.

I det andra exempelfallet vinner Lag B både första och andra settet med $25-27$, och resultatet av matchen blir alltså $0-2$.

Notera att enbart det första exempelfallet skulle kunna förekomma i testfallsgrupp $1$.
