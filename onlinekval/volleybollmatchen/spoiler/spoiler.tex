\documentclass{article}

\usepackage{listings}
\usepackage{amsmath,amssymb,amsfonts,amsthm}
\usepackage[utf8]{inputenc}

\begin{document}

\section*{Lösningsförslag Volleybollmatchen}

Här gäller det bara att simulera matchen. Vi håller koll på respektive lags poäng och går igenom poängen en för en. Ifall vi ser bokstaven texttt{A} lägger vi till ett på Algoritmikernas poäng, och ifall vi ser bokstaven texttt{B} lägger vi till ett på Bäverbusarnas poäng. Så fort vi uppdaterat poängen kollar vi ifall något av lagen vunnit settet, vilket görs genom att kolla ifall något lags poäng är större eller lika med winScore och dessutom minst två poäng mer än motståndarlaget. Värdet på variablen winScore ändrar vi beroende på vilket set som spelas just nu. Vi håller koll på hur många set respektive lag vunnit, och när något lag nåt 2 vunna set skriver vi ut resultatet. 

\lstinputlisting{volleybollmatchen.cpp}

\end{document}