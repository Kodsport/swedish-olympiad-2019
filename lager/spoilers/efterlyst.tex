\section*{Efterlyst}

Låt oss kalla de givna noderna för $\textit{speciella}$ noder.

\subsection*{Delpoäng 1}

Låt $p$ vara den minsta (längst till vänster) av de speciella noderna, och låt $q$ vara det största.
Nu kommer alla platser $r$ som uppfyller $r \leq p$ eller $r \geq q$ vara möjliga svar, så det är bara att printa ut dem.
Se upp med fallet $k = 1$.

\subsection*{Delpoäng 2}

Den här delpoängsnivån hjälper inte så mycket för att ta de högre nivåerna, så här är en kortfattad beskrivning:

När grafen är ett träd så spelar inte vikterna på kanterna någon roll eftersom det bara finns en väg mellan varje par av noder.
Om vi rotar trädet i en av de speciella noderna så kommer vägens ändpunkter inte ha några andra speciella noder i sitt subträd.
Detta gäller i alla fall utom när roten vi valde är en av ändpunkterna.

När vi har rett ut de olika fallen blir svaret alla noder som är i någon av ändpunkternas subträd.
Efter det återstår det bara att kolla att svaret är giltigt genom att kolla att alla speciella noder är på vägen mellan ändpunkterna.
Jobbigt att implementera, men ger en del poäng.

\subsection*{Delpoäng 3, 4, 5}

Här finns det lite olika lösningar som ger olika mycket poäng. En viktig observation som vi kommer behöva är att om 
$$b_1 \rightarrow b_2 \rightarrow \dots \rightarrow b_r$$
är en kortaste väg mellan $b_1$ och $b_r$, så är 
$$b_i \rightarrow \dots \rightarrow b_j$$
en kortaste väg mellan $b_i$ och $b_j$.
Detta innebär att om vi letar efter en kortaste väg till $y$ som går igenom alla speciella noder, så kan vi anta att vägen börjar vid en av de speciella noderna.

Låt oss testa en av de speciella noderna som startpunkt.
Gör en Dijkstra från $y$ för att räkna ut avstånden $d(y,i)$ till alla andra noder.
De kortaste vägarna bildar en slags riktad acyklisk graf.
Vi kan tänka oss att en riktad kant går mellan $i$ och $j$ om $d(y,i) + w_{ij} = d(y,j)$.
Nu behöver vi bara hitta alla noder sådana att det finns en väg i den riktade grafen som går igenom alla speciella noder och hamnar på $y$.
Det här går att göra i linjär tid, så totalt blir komplexiteten $O(km\log(n))$.

\subsection*{Delpoäng 6}

I förra delpoängsnivån löste vi problemet i $O(km\log(n))$ genom att kolla alla möjliga startpunkter.
Den huvudsakliga observationen för att lösa hela problemet är att det bara finns två speciella punkter som vi behöver kolla.
Låt oss säga att det finns en kortaste väg som går igenom alla speciella noder i ordningen $a_1, \dots, a_k$ och slutar på någon nod $y$.
Då är ju avståndet mellan $a_i$ och $a_j$ samma som avståndet på vägen.
Detta innebär att $a_1$ och $a_k$ är de speciella punkter som är längst ifrån varandra.
Ändpunkterna är alltså samma för alla möjliga vägar, så vi kan lösa problemet genom att köra $O(km\log(n))$-lösningen fast bara för de två ändpunkterna.

Nu återstår bara att faktiskt hitta ändpunkterna.
Det antagligen enklaste sättet att göra detta är att göra en Dijkstra från en godtycklig speciell nod, och se vilken av de andra speciella noderna som är längst bort.
Denna nod $a_1$ måste vara en av ändpunkterna, och då kan vi hitta den andra ändpunkten genom att göra samma sak från $a_1$.
Komplexiteten är $O(m\log(n))$.
