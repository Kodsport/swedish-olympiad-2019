\section*{E - Robottävlingen}
Att hitta maximala antalet är enkelt, det är bara att för varje ruta i rutnätet ta minimumvärdet av kolumnvärdet och radvärdet.

Att hitta minimala antalet är svårare. Vi börjar med att betrakta hur många femmor vi måste sätta ut, om vi vill ha så få som möjligt.
Vi måste såklart minst lägga en femma i varje kolumn som har värdet 5 (hur skulle kolumnen fått värdet 5 om den inte innehåller en femma?),
och vi måste på samma sätt även lägga en femma i varje rad.
Om $x_5$ är antalet rader med värdet 5 och $y_5$ antalet kolumner med värdet 5 behöver vi därför minst $max(x_5,y_5)$ stycken femmor.
Så länge det finns både någon 5-kolumn utan femma i sig och en 5-rad som inte innehåller en femma så kan vi placera en femma i den rutan som 
kolumnen och raden delar. Ifall vi bara har en 5-rad så kan vi placera ut en femma i den raden i någon kolumn som redan har en femma i sig,
och på samma sätt kan vi göra om vi bara har en 5-kolumn. Alltså går $max(x_5,y_5)$ antal femmor alltid att uppnå.

Samma argument kan vi använda för 4,3 och 2 får att bevisa att minsta antal fyror, treor och tvåor som går att uppnå är $max(x_4,y_4)$, $max(x_3,y_3)$ och $max(x_2,y_2)$ respektive.
Skillnaden är att man måste notera att när man till exempel har en 4-rad som man måste placera en fyra i, men inga 4-kolumner,
så kan vi placera fyran i någon av 5-kolumnerna istället. Det kommer alltid antingen finnas en 4-kolumn eller 5-kolumn att placera i,
eftersom det måste enligt raden vi jobbar med finnas en fyra någonstans, så någon kolumns maxvärde är minst 4.

Alla rutor som vi inte redan placerat något i placerar vi ettor i.

Notera att i lösningen nedan placerar vi aldrig faktiskt ut något, utan vi bara räknar direkt ut vad summan blir om vi hade placerat ut det.

\lstinputlisting{robottavlingen.cpp}