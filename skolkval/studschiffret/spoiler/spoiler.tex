\documentclass{article}

\usepackage{listings}
\usepackage{amsmath,amssymb,amsfonts,amsthm}
\usepackage[utf8]{inputenc}

\lstset{language=c++,breaklines=true}
\lstset{literate=%
{Ö}{{\"O}}1
{Ä}{{\"A}}1
{Å}{{\AA}}1
{ä}{{\"a}}1
{ö}{{\"o}}1
{å}{{\aa}}1
}

\begin{document}
\section*{Lösningsförslag Studschiffret}
Först simulerar vi studsandet och placerar ut talen $0$ till $str.size()-1$ i rutnätet.
Därefter läser vi av rutnätet rad för rad och lägger talen i en lista $encNum$.
Denna listan beskriver nu för varje bokstav i den krypterade strängen vilket index
den bokstaven har i den ursprungliga strängen,
så nu går vi helt enkelt igenom alla bokstäver i den kryptarade strängen
 och placerar varje bokstav på rätt plats i den ursprungliga strängen.

\lstinputlisting{studschiffret.cpp}

\end{document}