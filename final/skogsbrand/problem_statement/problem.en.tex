\def\version{jury-2}
\problemname{Forest Fire}
\noindent
A fire has broken out in the integer forest!
The integer forest consists of an infite two-dimensional plane where there is a tree at each point with integer coordinates.
Currently, $N$ of these trees are on fire.
Each minute, the fire spreads from every burning tree to its four neighbours (the trees immediately north, east, west and south).
To stop the fire, the fire marshals have chopped down $M$ trees.
A chopped-down tree can not catch fire, so these points act like a kind of wall to the flames.
You are interested in the damage that the fire will cause.
In $T$ minutes, there will be a great rain storm that will put out the entire fire.
Therefore, you want to know how many trees are on fire after $T$ minutes.

\section*{Input}
The first line contains three integers:
\begin{itemize}
        \item the number of burning trees $N$ ($1 \leq N \leq 100$),
        \item the number of chopped-down trees $M$ ($0 \leq M \leq 100$),
        \item and the number of minutes to the great rain storm $T$ ($0 \leq T \leq 10^9$).
\end{itemize}

The following $N$ lines contains two integers $x_i,y_i$ ($0 \leq x_i, y_i < 10^5$), the coordinates of the burning trees.

The following $M$ lines contains two integers $x_i,y_i$ ($0 \leq x_i, y_i < 10^5$), the coordinates of the chopped-down trees.

No two trees, either burning or chopped down, will have the same coordinates.

\section*{Output}
Output a single integer, the number of trees that are on fire after $T$ minutes.

\section*{Scoring}
Your solution will be tested on a set of test case groups.
To get the points for a group, you need to pass all the test cases in the group.

\noindent
\begin{tabular}{| l | l | l |}
\hline
Group & Points & Constraints \\ \hline
$1$   & $5$        & $N = 1$, $M = 0$ \\ \hline
$2$   & $9$        & $T \leq 100$, $M = 0$, $x_i, y_i < 100$\\ \hline
$3$   & $10$       & $T \leq 400$, $x_i, y_i < 100$ \\ \hline
$4$   & $11$       & $M = 0$, $x_i, y_i < 100$ \\ \hline
$5$   & $21$       & $x_i, y_i < 100$ \\ \hline
$6$   & $15$       & $M = 0$ \\ \hline
$7$   & $29$       & No further constraints \\ \hline
\end{tabular}

\section*{Explanations of samples}
In sample $1$ the fire spreads to $(0,2)$, $(1,3)$ and $(2,2)$.
In total, there will be $4$ burning trees.

In sample $2$ the only burning tree is surrounded by chopped-down trees.
There is thus only a single burning tree, despite $T$ being large.

In sample $3$ the fire does not have time to spread since $T = 0$, so the answer is $4$.
